%-------------------------
% Resume in Latex
% Author : Sourabh Bajaj
% License : MIT
%------------------------

\documentclass[letterpaper,11pt]{article}

\usepackage{latexsym}
\usepackage[empty]{fullpage}
\usepackage{titlesec}
\usepackage{marvosym}
\usepackage[usenames,dvipsnames]{color}
\usepackage{verbatim}
\usepackage{enumitem}
\usepackage[hidelinks]{hyperref}
\usepackage{fancyhdr}
\usepackage[english]{babel}
\usepackage{tabularx}
\usepackage{ragged2e}
\usepackage{mathptmx}

\input{glyphtounicode}

% \renewcommand{\rmdefault}{ptm}

\pagestyle{fancy}
\fancyhf{} % clear all header and footer fields
\fancyfoot{}
\renewcommand{\headrulewidth}{0pt}
\renewcommand{\footrulewidth}{0pt}

% Adjust margins
\addtolength{\oddsidemargin}{-0.5in}
\addtolength{\evensidemargin}{-0.5in}
\addtolength{\textwidth}{1in}
\addtolength{\topmargin}{-.5in}
\addtolength{\textheight}{1.0in}

\urlstyle{same}

\raggedbottom
\raggedright
\setlength{\tabcolsep}{0in}

% Sections formatting
\titleformat{\section}{
  \vspace{-4pt}\scshape\raggedright\large\fontfamily{phv}\selectfont
}{}{0em}{}[\color{black}\titlerule \vspace{-5pt}]

% Ensure that generate pdf is machine readable/ATS parsable
\pdfgentounicode=1

%-------------------------
% Custom commands
\newcommand{\resumeItem}[2]{
  \item\small{
    \textbf{#1}{: #2 \vspace{-2pt}}
  }
}

% Just in case someone needs a heading that does not need to be in a list
\newcommand{\resumeHeading}[4]{
    \begin{tabular*}{0.99\textwidth}[t]{l@{\extracolsep{\fill}}r}
      \textbf{#1} & #2 \\
      \textit{\small#3} & \textit{\small #4} \\
    \end{tabular*}\vspace{-5pt}
}

\newcommand{\resumeSubheading}[4]{
  \item[] % REMOVEBULLET: Use to reduce margin from bullet point
  % \item
    \begin{tabular*}{1.0\textwidth}[t]{l@{\extracolsep{\fill}}r} % REMOVEBULLET: Use to reduce margin from bullet point
    % \begin{tabular*}{0.97\textwidth}[t]{l@{\extracolsep{\fill}}r}
      \textbf{#1} & #2 \vspace{-2pt}\\ 
      \textit{\small#3} & \textit{\small #4} \\
    \end{tabular*}\vspace{-8.5pt}
}

\newcommand{\resumeSubSubheading}[2]{
    \begin{tabular*}{1.0\textwidth}{l@{\extracolsep{\fill}}r} % REMOVEBULLET: Use to reduce margin from bullet point
    % \begin{tabular*}{0.97\textwidth}{l@{\extracolsep{\fill}}r}
      \textit{\small#1} & \textit{\small #2} \\
    \end{tabular*}\vspace{-8.5pt}
}

\newcommand{\resumeEducationSubheading}[4]{
  \item \begin{tabular*}{0.97\textwidth}[t]{l@{\extracolsep{\fill}}r}
      \textbf{#1} $|$ \textit{\small#3} & \textit{\small #4} $|$ #2
    \end{tabular*}\vspace{-10pt}
}


\newcommand{\resumeSubItem}[2]{\resumeItem{#1}{#2}\vspace{-4pt}}

% \newcommand{\resumePaperItem}[2]{
%   \item\small{
%     {#1}{#2\vspace{-4pt}}
%   }
% }
% \newcommand{\resumePaperItem}[2]{
%   \begin{tabular*}{0.97\textwidth}{l@{\extracolsep{\fill}}r}
%     \textit{\small#1} & \textit{\small #2} \\
%   \end{tabular*}\vspace{-5pt}}

\newcommand{\datedline}[2]{%
  {\par #1 \hfill #2 \par}%
}

% \setlist[itemize]{leftmargin=*}

% \newcommand{\resumePaperItem}[2]{
%   \vspace{-1pt}\item
%     \begin{tabular*}{0.97\textwidth}[t]{l@{\extracolsep{\fill}}r}
%       \textbf{\small#1} & #2
%     \end{tabular*}\vspace{-10pt}
% }


\renewcommand{\labelitemii}{$\circ$}

\newcommand{\resumeSubHeadingListStart}{\begin{itemize}[leftmargin=0pt]} % REMOVEBULLET: Use to reduce margin from bullet point
  % \newcommand{\resumeSubHeadingListStart}{\begin{itemize}[leftmargin=0pt]}
\newcommand{\resumeSubHeadingListEnd}{\end{itemize}}
\newcommand{\resumeItemListStart}{\begin{justify}\begin{itemize}[leftmargin=*]}
\newcommand{\resumeItemListEnd}{\end{itemize}\end{justify}\vspace{-5pt}}

\newcommand{\resumeItemTitleAndStart}[1]{
          \item\small{
          \textbf{#1\vspace{-2pt}}
          }
          \begin{itemize}[topsep=0pt, leftmargin=*]
          }


%-------------------------------------------
%%%%%%  CV STARTS HERE  %%%%%%%%%%%%%%%%%%%%%%%%%%%%


\begin{document}

%----------HEADING-----------------
\begin{tabular*}{\textwidth}{l@{\extracolsep{\fill}}r}
  \textbf{\href{https://www.linkedin.com/in/maxence-bouvier/}{\Large\fontfamily{phv}\selectfont Maxence Bouvier, PhD}} & Email : \href{mailto:maxence.bouvier.pro@gmail.com}{maxence.bouvier.pro@gmail.com}\\
  \href{https://www.linkedin.com/in/maxence-bouvier/}{LinkedIn} $|$ \href{https://scholar.google.com/citations?user=AeceaFAAAAAJ&hl=en}{GoogleScholar} & Mobile : \href{tel:+41782107194}{+41-78-210-71-94} \\
\end{tabular*}



%-----------CENTERED TEXT-----------------
\begin{justify}
  \small{I discovered AI's potential six years ago at IBM, inspiring my Ph.D. studies to build energy-efficient machine learning vision systems. At Sony, I have been leading a project on multimodal transformer models for low power vision systems. I mentored many successful students during their thesis and internships, which helped me patent several ideas. I'm now eager to create AI solutions that will impact millions.}
\end{justify}\vspace{-10pt}


%-----------EXPERIENCE-----------------
% TODO: find to way to talk about internship supervision
\section{Experience}
  \resumeSubHeadingListStart
    \resumeSubheading
      {SONY}{Zurich, Switzerland}
      {Senior AI Research Engineer}{Aug 2023 - Present}
      \resumeItemListStart
        \resumeItemTitleAndStart{Sparsity Exploitation in Transformers}
          \item{Engineered an asynchronous PointNet-based embedding, enabling continuous spatio-temporal data conversion into dense tensors for seamless, ongoing feeding of Transformer models - (\textit{1 paper, 1 patent.})\vspace{-2pt}}
          \item{Designed an NPU-compatible sparse scaled-dot-product-attention per-block module for highly efficient sparse attention in Transformers, achieving more than 50\% FLOPs reduction during inference and higher accuracy.\vspace{-2pt}}
        \end{itemize}
        
        \resumeItem{SLAM Enhanced AI Training}
          {Enhanced performance by incorporating a cutting-edge SLAM pipeline for multi-modal training process, achieving 6x faster model convergence and a 15\% boost in accuracy.}
        % \resumeItem{FPGA Design Innovator - FPGA Design Lead}
          % {Implemented a state-of-the-art strategy in FPGA for per-block sparse attention, yielding over 20x increase in energy efficiency compared to standard GPU processes.}
        % \resumeItem{Technical Supervision}
        %   {Mentored interns in AI-based visual odometry and cost-effective data embedding for event-based cameras.}
        % \resumeItem{Scrum Master}
        %   {Managed project initiatives as a repository maintainer and Scrum Master, ensuring efficient and streamlined project progress.}
        % \resumeItem{Communication}
          % {Facilitated global knowledge sharing through patents, papers, thesis contributions, speaking engagements, and extensive internal presentations.}
               
      \resumeItemListEnd
          
    \resumeSubSubheading
      {AI Research Engineer}{June 2022 - Aug 2023}
      \resumeItemListStart
        \resumeItemTitleAndStart{SW/HW Co-Design Automation with Neural Architecture Search}
          \item{Built an AI-driven, Hardware-Aware Neural Architecture Search framework. Enabled to automatically reduce model to 8\% FLOP cost while losing only 4\% relative accuracy.\vspace{-2pt}}
          \item{Integrated a Design Space Exploration software in the NAS loop to estimate energy and latency of model execution.\vspace{-2pt}}
        \end{itemize}
        \resumeItem{Transformer Hardware Acceleration Survey}
          % {Conducted a Transformer acceleration study, guiding Sony's AI advancements; featured in CTO's strategic report. - (Attended ISSCC23)}
          {Conducted a literature study, featured in CTO's strategic report.}
        \resumeItemTitleAndStart{Vision Transformer for Image Generation}
          \item{Implemented an AI model leveraging CNN and Transformer architectures to realize advanced frame generation. - (\textit{1 patent.})\vspace{-2pt}}
          \item{Built a live demo of the model, from image sensor to application. This led to 2 major collaborations with other teams.\vspace{-2pt}}
          \item{Converted the model as an API to simplify sharing across teams and projects.\vspace{-2pt}}
        \end{itemize}
        \resumeItem{Software Maintainer}
          {Responsible for the CI of a few Python libraries shared among teams.}
      \resumeItemListEnd

    \resumeSubheading
      {STMicroelectronics}{Grenoble, France}
      {Hardware Design Engineer}{Apr 2021 - May 2022}
      \resumeItemListStart
        \resumeItemTitleAndStart{CPU Design and Automation}
          % {Built Trace and Debug subsystem for ARM's Armv9-A SoC from scratch and created a Python library for automated component assembly.}
          \item{Created a toolbox to automate component assembly of the Trace and Debug subsystem with ARM's Armv9-A SoC modules.\vspace{-2pt}}
          \item{Developed an RTL generator for STM32 MPU SoC, streamlining the design of a multi-clock domain reset and clock control for over 300 peripherals.\vspace{-2pt}}
        \end{itemize}
        \resumeItem{CPU Benchmarking}
          {Conducted CoreMark benchmarking on a multi-core MPU SoC, highlighting significant performance gains (up to 6x) through compiler updates.}
        % \resumeItem{Training}
          % {Gained specialized training on the latest ARM IPs, focusing on V9 architecture and AMBA5 protocols, enhancing technical proficiency in cutting-edge CPU architecture.}
      \resumeItemListEnd

    \resumeSubheading
      {CEA LETI}{Grenoble, France}
      {Doctoral Researcher on AI and Hardware Design}{Apr 2018 - Apr 2021}
      \resumeItemListStart
        \resumeItem{Neuromorphic Hardware Analysis}
          {Conducted a comprehensive bibliographic study on scalable, multi-chip, distributed neuromorphic hardware, leading to a widely cited publication in ACM JETC. - (\textit{1 paper}.)}
        \resumeItem{ULP NPU Design}
          % {Built (RTL design, synthesis and layout) a 28nm FDSOI ultra-low power sparse AI accelerator, setting energy efficiency records (2.86pJ/OP) and enabling seamless Event-Based pixel grid integration. - (\textit{1 paper, 2 patents.})}
          {Built (RTL design, synthesis and layout) an ultra-low power sparse AI accelerator, setting energy efficiency records (2.86pJ/OP in 28nm) and enabling seamless integration for 3D-stacked imagers. - (\textit{1 paper, 2 patents.})}
        \resumeItem{EB VIO/SLAM Pipeline and Object Detection Innovation}
          {Developed an Event-Based VIO/SLAM pipeline with ego-motion compensation, leading to a solution for detecting moving objects. - (\textit{1 patent.})}
        % \resumeItem{Specialized Dataset Development}
        %   {Generated two unique datasets, one with Blender and another using CeleX-V imager, improving data accuracy and applicability.}
      \resumeItemListEnd

    \resumeSubheading
      {IBM Research}{Yorktown Heights, NY, USA}
      {Intern Hardware Engineer}{Feb 2017 - Aug 2017}
      \resumeItemListStart
        \item\small{Automated wafer-scale memory device characterization, reducing execution time from days to hours.\vspace{-2pt}}
        \item\small{Contributed to the optimization of PCM technologies for Compute-in-Memory-based AI acceleration. - (\textit{1 paper, 1 patent.})\vspace{-2pt}}
      \resumeItemListEnd

      % {UPENN}
      % https://react.seas.upenn.edu/penns-react-team-welcomes-four-grenoble-students-for-summer-2016/
      % https://react.seas.upenn.edu/penns-react-team-welcomes-four-grenoble-students-for-summer-2016/

  \resumeSubHeadingListEnd


%-----------Patents and Publications-----------------
% \section{Notable Publications}
%   % \resumeItemListStart
%     \datedline{Scalable Pitch-Constrained Neural Processing Unit for 3D Integration with Event-Based Imagers}{2021}
%     % \datedline{Advanced 3D Technologies and Architectures for 3D Smart Image Sensors}{2021}
%     \datedline{Spiking Neural Network Hardware Implementations and Challenges: A Survey}{2019 - 200 cites}
%     \datedline{On-Chip Trainable 1.4M 6T2R PCM Synaptic Array with 1.6K Stochastic LIF Neurons for Spiking RBM}{2019}
  % \resumeItemListEnd
%-----------PROJECTS-----------------
% \section{Projects}
%   \resumeSubHeadingListStart
%     \resumeSubItem{QuantSoftware Toolkit}
%       {Open source python library for financial data analysis and machine learning for finance.}
%     \resumeSubItem{Github Visualization}
%       {Data Visualization of Git Log data using D3 to analyze project trends over time.}
%     \resumeSubItem{Recommendation System}
%       {Music and Movie recommender systems using collaborative filtering on public datasets.}
%     \resumeSubItem{Mac Setup}
%       {Book that gives step by step instructions on setting up developer environment on Mac OS.}
%   \resumeSubHeadingListEnd

%
%--------PROGRAMMING SKILLS------------
\section{Skills}
\begin{itemize}[leftmargin=*, topsep=0pt]
   \item
     \textbf{Languages}{: Python, C, C++, MATLAB, SystemVerilog, VHDL, LaTeX}\vspace{-10pt}
   \item
     \textbf{Libraries}{: PyTorch, MLFlow, ONNX, Numpy, Pandas, ROS, OpenCV, CUDA}\vspace{-10pt}
   \item
     \textbf{Softwares}{: Git, GitLab CI, EDA (Cadence, Synopsys, Mentor), VSCode, PowerPoint Expert}\vspace{-10pt}
  \end{itemize}

% Programming and Development
% Python, Pytorch; Transformers, CNNs; OpenCV, ONNX; VSCode, Git, Linux, ROS
% Hardware and System Design
% RTL: VHDL, SystemVerilog; Image Sensors; Compute-in-Memory; Xilinx FPGAs; ARM V9 Arch., AMBA5
% Tools and Workflow
% IP-XACT, EDA Tools; Full IC Tapeout Flow
% Soft Skills and Miscellaneous
% Task Automator; Great Tutorials Provider; Fantastic PowerPoint Maker



%-----------EDUCATION-----------------
\section{Education}
  \begin{itemize}[leftmargin=*, topsep=0pt]
    \resumeEducationSubheading
      {Grenoble Alpes University}{Grenoble, France}
      {Ph.D. in Computer Science}{Apr 2018 -- Apr 2021}
    \resumeEducationSubheading
      {EPFL}{Lausanne, Switzerland}
      {"M.Eng. in Electronics (Highest Honors)"}{Sep 2015 -- Sep 2017}
    \resumeEducationSubheading
      {Grenoble Institute of Technology}{Grenoble, France}
      {B.Eng. in Electronics}{Sep 2012 -- Sep 2015}
  \end{itemize}



%-------------------------------------------
\end{document}
