%-------------------------
% Resume in Latex
% Author : Sourabh Bajaj
% License : MIT
%------------------------

\documentclass[letterpaper,11pt]{article}

\usepackage{latexsym}
\usepackage[empty]{fullpage}
\usepackage{titlesec}
\usepackage{marvosym}
\usepackage[usenames,dvipsnames]{color}
\usepackage{verbatim}
\usepackage{enumitem}
\usepackage[hidelinks]{hyperref}
\usepackage{fancyhdr}
\usepackage[english]{babel}
\usepackage{tabularx}
\usepackage{ragged2e}
\usepackage{mathptmx}

\input{glyphtounicode}

% \renewcommand{\rmdefault}{ptm}

\pagestyle{fancy}
\fancyhf{} % clear all header and footer fields
\fancyfoot{}
\renewcommand{\headrulewidth}{0pt}
\renewcommand{\footrulewidth}{0pt}

% Adjust margins
\addtolength{\oddsidemargin}{-0.5in}
\addtolength{\evensidemargin}{-0.5in}
\addtolength{\textwidth}{1in}
\addtolength{\topmargin}{-.5in}
\addtolength{\textheight}{1.0in}

\urlstyle{same}

\raggedbottom
\raggedright
\setlength{\tabcolsep}{0in}

% Sections formatting
\titleformat{\section}{
  \vspace{-4pt}\scshape\raggedright\large\fontfamily{phv}\selectfont
}{}{0em}{}[\color{black}\titlerule \vspace{-5pt}]

% Ensure that generate pdf is machine readable/ATS parsable
\pdfgentounicode=1

%-------------------------
% Custom commands
\newcommand{\resumeItem}[2]{
  \item\small{
    \textbf{#1}{: #2 \vspace{-2pt}}
  }
}

% Just in case someone needs a heading that does not need to be in a list
\newcommand{\resumeHeading}[4]{
    \begin{tabular*}{0.99\textwidth}[t]{l@{\extracolsep{\fill}}r}
      \textbf{#1} & #2 \\
      \textit{\small#3} & \textit{\small #4} \\
    \end{tabular*}\vspace{-5pt}
}

\newcommand{\resumeSubheading}[4]{
  \item[] % REMOVEBULLET: Use to reduce margin from bullet point
  % \item
    \begin{tabular*}{1.0\textwidth}[t]{l@{\extracolsep{\fill}}r} % REMOVEBULLET: Use to reduce margin from bullet point
    % \begin{tabular*}{0.97\textwidth}[t]{l@{\extracolsep{\fill}}r}
      \textbf{#1} & #2 \vspace{-2pt}\\ 
      \textit{\small#3} & \textit{\small #4} \\
    \end{tabular*}\vspace{-8.5pt}
}

\newcommand{\resumeSubSubheading}[2]{
    \begin{tabular*}{1.0\textwidth}{l@{\extracolsep{\fill}}r} % REMOVEBULLET: Use to reduce margin from bullet point
    % \begin{tabular*}{0.97\textwidth}{l@{\extracolsep{\fill}}r}
      \textit{\small#1} & \textit{\small #2} \\
    \end{tabular*}\vspace{-8.5pt}
}

\newcommand{\resumeEducationSubheading}[4]{
  \item \begin{tabular*}{0.97\textwidth}[t]{l@{\extracolsep{\fill}}r}
      \textbf{#1} $|$ \textit{\small#3} & \textit{\small #4} $|$ #2
    \end{tabular*}\vspace{-10pt}
}


\newcommand{\resumeSubItem}[2]{\resumeItem{#1}{#2}\vspace{-4pt}}

% \newcommand{\resumePaperItem}[2]{
%   \item\small{
%     {#1}{#2\vspace{-4pt}}
%   }
% }
% \newcommand{\resumePaperItem}[2]{
%   \begin{tabular*}{0.97\textwidth}{l@{\extracolsep{\fill}}r}
%     \textit{\small#1} & \textit{\small #2} \\
%   \end{tabular*}\vspace{-5pt}}

\newcommand{\datedline}[2]{%
  {\par #1 \hfill #2 \par}%
}

% \setlist[itemize]{leftmargin=*}

% \newcommand{\resumePaperItem}[2]{
%   \vspace{-1pt}\item
%     \begin{tabular*}{0.97\textwidth}[t]{l@{\extracolsep{\fill}}r}
%       \textbf{\small#1} & #2
%     \end{tabular*}\vspace{-10pt}
% }


\renewcommand{\labelitemii}{$\circ$}

\newcommand{\resumeSubHeadingListStart}{\begin{itemize}[leftmargin=0pt]} % REMOVEBULLET: Use to reduce margin from bullet point
  % \newcommand{\resumeSubHeadingListStart}{\begin{itemize}[leftmargin=0pt]}
\newcommand{\resumeSubHeadingListEnd}{\end{itemize}}
\newcommand{\resumeItemListStart}{\begin{justify}\begin{itemize}[leftmargin=*]}
\newcommand{\resumeItemListEnd}{\end{itemize}\end{justify}\vspace{-5pt}}

\newcommand{\resumeItemTitleAndStart}[1]{
          \item\small{
          \textbf{#1\vspace{-2pt}}
          }
          \begin{itemize}[topsep=0pt, leftmargin=*]
          }


%-------------------------------------------
%%%%%%  CV STARTS HERE  %%%%%%%%%%%%%%%%%%%%%%%%%%%%


\begin{document}

%----------HEADING-----------------
\begin{tabular*}{\textwidth}{l@{\extracolsep{\fill}}r}
  \textbf{\href{https://www.linkedin.com/in/maxence-bouvier/}{\Large\fontfamily{phv}\selectfont Maxence Bouvier, PhD}} & Email : \href{mailto:maxence.bouvier.pro@gmail.com}{maxence.bouvier.pro@gmail.com}\\
  \href{https://www.linkedin.com/in/maxence-bouvier/}{LinkedIn} $|$ \href{https://scholar.google.com/citations?user=AeceaFAAAAAJ&hl=en}{GoogleScholar} & Mobile : \href{tel:+41782107194}{+41-78-210-71-94} \\
\end{tabular*}




%-----------CENTERED TEXT-----------------
\begin{justify}
  \small{AI \& HW Research Scientist with 8+ years of experience in ML and Chip Design. Currently building LLM-driven circuit design automation at Arago (AIMÉE team). Expertise spans LLM-based EDA automation, analog/digital circuit optimization, Bayesian design space exploration, energy-efficient HW accelerators, and ML infrastructure tooling.}
\end{justify}\vspace{-10pt}


%-----------EXPERIENCE-----------------
% TODO: find to way to talk about internship supervision
\section{Experience}
  \resumeSubHeadingListStart
    \resumeSubheading
      {Arago}{Zurich, Switzerland}
      {AI for Chip Design}{Nov 2025 - Present}
      \resumeItemListStart
        \resumeItemTitleAndStart{LLM-Driven Circuit Design Automation (AIMÉE)}
          \item{Built \textbf{circuit\_explorer}: FunSearch-inspired LLM exploration engine for analog/digital circuits. LLM generates Python circuit constructors, evaluated via ngspice/Optuna two-stage optimization. Supports Claude, OpenAI, HuggingFace, and local llama.cpp backends.\vspace{-2pt}}
          \item{Developed \textbf{virtuoso\_adapter}: Python automation framework for Cadence Virtuoso with bidirectional SKILL bridge. Enables automated ADC characterization (ENOB/SFDR/SNR), waveform analysis (PSF to NumPy), and remote Spectre simulation.\vspace{-2pt}}
          \item{Created \textbf{ngspice\_playground}: Docker-based EDA environments with analog (NGSPICE + SKY130 PDK) and digital (Yosys + OpenROAD + ASAP7) toolchains, GPU-accelerated ML, and CI/CD automation.\vspace{-2pt}}
        \end{itemize}
        \resumeItem{Memory Configuration Optimization}
          {Developed \textbf{memcfg\_opt}: Bayesian optimization tool (Optuna) for ARM SRAM design. Multi-objective optimization (area, timing, power), 10-100x faster than grid search, with 1027+ tests and 95\% coverage.\vspace{-2pt}}
        \resumeItemTitleAndStart{AI/ML Infrastructure \& Tooling}
          \item{Built \textbf{hf\_quantizer}: CLI tool for 4-bit model quantization using SINQ (Sinkhorn-Normalized Quantization) with automated HuggingFace Hub deployment and HumanEval verification.\vspace{-2pt}}
          \item{Created \textbf{mcp\_kiwi}: Model Context Protocol server connecting Claude to internal wiki (Outline), with LRU caching, rate limiting, YunoHost SSO, and full document/collection CRUD.\vspace{-2pt}}
          \item{Developed internal Claude Code plugin marketplace for team-wide AI tool distribution and management.\vspace{-2pt}}
        \end{itemize}
      \resumeItemListEnd

    \resumeSubheading
      {Huawei}{Zurich, Switzerland}
      {AI \& HW Research Scientist}{May 2024 - Oct 2025}
      \resumeItemListStart
        \resumeItem{Team Leader}
          {Built and led a multidisciplinary team of experts to develop innovative solutions for reducing switching activity and power consumption in Huawei's GPUs.\vspace{-2pt}}
        \resumeItemTitleAndStart{ML for Chip Design}
        \item{Designed and developed an automated ML-based framework for iterative generation of multiplier circuits with reduced switching activity and lower power consumption. The full pipeline was up and running in less than three months, working alone.\vspace{-2pt}}
        \item{Deployed synthesis and simulation of millions of designs using containerized (Docker), open source EDA tools on a multi-node SLURM cluster. Adapted the framework for commercial EDA tools.\vspace{-2pt}}
        \item{Proved that Network Inversion is superior to other state of the art method for design generation (and design space exploration).\vspace{-2pt}}
        \item{Results: we found multiplier encodings that reduce power consumption by approximately 10\% compared to conventional two's complement implementations. (\textit{1 paper.})\vspace{-2pt}}
        \end{itemize}
        \resumeItem{ML for Advanced Synthesis}
        {Developed a predictor-driven Logic Synthesis Optimization framework achieving up to 21\% QoR improvement and 14× faster execution. - (\textit{2 papers.})\vspace{-2pt}}
        \resumeItemTitleAndStart{Characterization of ML Workloads Acceleration}
        \item{Developed a simulation platform to accurately map (tiling and multi-core scheduling) tensor operations onto Huawei's Ascend "Cube" tensor accelerator.\vspace{-2pt}}
        \item{Exploited the simulator to benchmark tensor reshaping and vector reordering strategies, proposing novel software-level optimizations that effectively reduce power consumption.\vspace{-2pt}}
        \end{itemize}
        \resumeItem{Conference Attendance} {2025 DAC, 2025 International Conference on LLM-Aided Design (ICLAD).\vspace{-2pt}}
        % {Built a simulator that would map (tiling and multi-core scheduling) tensor operations onto a multi-dimensional tensor processing accelerator (Huawei's Ascend "Cube"). Used the simulator to characterize the energy efficiency of some tiling and scheduling patterns and proposed software-level solutions to reduce power consumption of the "Cube".\vspace{-2pt}}
        \resumeItemListEnd
        % \vspace{-10pt}
          
    \resumeSubheading
      {SONY}{Zurich, Switzerland}
      {Senior AI Research Engineer}{Aug 2023 - Apr 2024}
      \resumeItemListStart
        \resumeItemTitleAndStart{Sparsity Exploitation in Transformers}
          \item{Engineered an asynchronous PointNet-based embedding, enabling continuous spatio-temporal data conversion into dense tensors for seamless, continuous feeding of Transformer models. - (\textit{1 paper, 1 patent.})\vspace{-2pt}}
          \item{Designed an NPU-compatible, block-wise sparse scaled dot-product attention module for highly efficient flash attention in Transformers, achieving more than 50\% FLOPs reduction during inference and higher accuracy.\vspace{-2pt}}
        \end{itemize}
        
        \resumeItem{SLAM Enhanced AI Training}
          {Enhanced performance by incorporating a cutting-edge SLAM pipeline for multi-modal training process, achieving 6x faster model convergence and a 15\% accuracy improvement.}
        % \resumeItem{FPGA Design Innovator - FPGA Design Lead}
          % {Implemented a state-of-the-art strategy in FPGA for per-block sparse attention, yielding over 20x increase in energy efficiency compared to standard GPU processes.}
        % \resumeItem{Technical Supervision}
        %   {Mentored interns in AI-based visual odometry and cost-effective data embedding for event-based cameras.}
        % \resumeItem{Scrum Master}
        %   {Managed project initiatives as a repository maintainer and Scrum Master, ensuring efficient and streamlined project progress.}
        % \resumeItem{Communication}
          % {Facilitated global knowledge sharing through patents, papers, thesis contributions, speaking engagements, and extensive internal presentations.}
               
      \resumeItemListEnd
          
    \resumeSubSubheading
      {AI Research Engineer}{June 2022 - Aug 2023}
      \resumeItemListStart
        \resumeItemTitleAndStart{SW/HW Co-Design Automation with Neural Architecture Search}
          \item{Built an AI-driven, Hardware-Aware Neural Architecture Search framework. Reduced model FLOP cost to 8\% of the original, with only a 4\% accuracy loss.\vspace{-2pt}}
          \item{Integrated a Design Space Exploration software in the NAS loop to estimate energy and latency of model execution.\vspace{-2pt}}
        \end{itemize}
        \resumeItem{Transformer Hardware Acceleration Survey}
          % {Conducted a Transformer acceleration study, guiding Sony's AI advancements; featured in CTO's strategic report. - (Attended ISSCC23)}
          {Conducted a literature study, featured in CTO's strategic report.}
        \resumeItemTitleAndStart{Vision Transformer for Image Generation}
          \item{Implemented an AI model leveraging CNN and Transformer architectures to realize advanced frame generation. - (\textit{1 patent.})\vspace{-2pt}}
          \item{Built a live demo of the model, from image sensor to application. This led to 2 major collaborations with other teams.\vspace{-2pt}}
          \item{Packaged the model as an API to simplify sharing across teams and projects.\vspace{-2pt}}
        \end{itemize}
        \resumeItem{Software Maintainer}
          {Responsible for the CI of a few Python libraries shared among teams.}
        \resumeItem{Conference Attendance} {2023 ISSCC.}

\resumeItemListEnd

\resumeSubheading
{STMicroelectronics}{Grenoble, France}
{Digital IC Design Engineer}{Apr 2021 - May 2022}
\resumeItemListStart
\resumeItemTitleAndStart{CPU Design and Automation}
% {Built Trace and Debug subsystem for ARM's Armv9-A SoC from scratch and created a Python library for automated component assembly.}
\item{Created a toolbox to automate component assembly of the Trace and Debug subsystem with ARM's Armv9-A SoC modules.\vspace{-2pt}}
\item{Developed an RTL generator for STM32 MPU SoC, streamlining the design of a multi-clock-domain reset and clock-control system for over 300 peripherals.\vspace{-2pt}}
\end{itemize}
\resumeItem{CPU Benchmarking}
{Conducted CoreMark benchmarking on a multi-core MPU SoC, highlighting significant performance gains (up to 6x) through compiler updates.}
% \resumeItem{Training}
% {Gained specialized training on the latest ARM IPs, focusing on V9 architecture and AMBA5 protocols, enhancing technical proficiency in cutting-edge CPU architecture.}
\resumeItem{Conference Attendance} {2021 ISSCC.}
      \resumeItemListEnd

    \resumeSubheading
      {CEA LETI}{Grenoble, France}
      {Doctoral Researcher on AI and Digital IC Design}{Apr 2018 - Apr 2021}
      \resumeItemListStart
        \resumeItem{Neuromorphic Hardware Survey}
          {Conducted a comprehensive literature review on scalable, distributed, multi-chip neuromorphic hardware, leading to a widely cited publication in ACM JETC. - (\textit{1 paper}.)}
        \resumeItem{ULP NPU Design}
          % {Built (RTL design, synthesis and layout) a 28nm FDSOI ultra-low-power sparse AI accelerator, setting energy efficiency records (2.86pJ/OP) and enabling seamless Event-Based pixel grid integration. - (\textit{1 paper, 2 patents.})}
          {Built (RTL design, synthesis and layout) an ultra-low-power sparse AI accelerator, setting energy efficiency records (2.86pJ/OP in 28nm) and enabling seamless integration for 3D-stacked imagers. - (\textit{1 paper, 2 patents.})}
        \resumeItem{EB VIO/SLAM Pipeline and Object Detection Innovation}
          {Developed an Event-Based VIO/SLAM pipeline with ego-motion compensation, leading to a solution for detecting moving objects. - (\textit{1 patent.})}
        \resumeItem{Conference Attendance} {2019 ISSCC, 2021 DAC.}
        % \resumeItem{Specialized Dataset Development}
        %   {Generated two unique datasets, one with Blender and another using CeleX-V imager, improving data accuracy and applicability.}
      \resumeItemListEnd

    \resumeSubheading
      {IBM Research}{Yorktown Heights, NY, USA}
      {Intern IC Design Engineer}{Feb 2017 - Aug 2017}
      \resumeItemListStart
        \item\small{Automated wafer-scale memory device characterization, reducing execution time from days to hours.\vspace{-2pt}}
        \item\small{Contributed to the optimization of PCM technologies for Compute-in-Memory-based AI acceleration. - (\textit{1 paper, 1 patent.})\vspace{-2pt}}
      \resumeItemListEnd

      % {UPENN}
      % https://react.seas.upenn.edu/penns-react-team-welcomes-four-grenoble-students-for-summer-2016/
      % https://react.seas.upenn.edu/penns-react-team-welcomes-four-grenoble-students-for-summer-2016/

  \resumeSubHeadingListEnd


%-----------Patents and Publications-----------------
% \section{Notable Publications}
%   % \resumeItemListStart
%     \datedline{Scalable Pitch-Constrained Neural Processing Unit for 3D Integration with Event-Based Imagers}{2021}
%     % \datedline{Advanced 3D Technologies and Architectures for 3D Smart Image Sensors}{2021}
%     \datedline{Spiking Neural Network Hardware Implementations and Challenges: A Survey}{2019 - 200 cites}
%     \datedline{On-Chip Trainable 1.4M 6T2R PCM Synaptic Array with 1.6K Stochastic LIF Neurons for Spiking RBM}{2019}
  % \resumeItemListEnd
%-----------PROJECTS-----------------
% \section{Projects}
%   \resumeSubHeadingListStart
%     \resumeSubItem{QuantSoftware Toolkit}
%       {Open source python library for financial data analysis and machine learning for finance.}
%     \resumeSubItem{Github Visualization}
%       {Data Visualization of Git Log data using D3 to analyze project trends over time.}
%     \resumeSubItem{Recommendation System}
%       {Music and Movie recommender systems using collaborative filtering on public datasets.}
%     \resumeSubItem{Mac Setup}
%       {Book that gives step by step instructions on setting up developer environment on Mac OS.}
%   \resumeSubHeadingListEnd

%
%--------PROGRAMMING SKILLS------------
\section{Skills}
\begin{itemize}[leftmargin=*, topsep=0pt, label=$\circ$]
   \item
     \textbf{Languages}{: Python, C/C++, SystemVerilog, SKILL, MATLAB, VHDL}\vspace{-10pt}
   \item
     \textbf{AI/ML}{: PyTorch, Transformers, LLM APIs (Claude, OpenAI), llama.cpp, Optuna, SINQ, MLFlow, ONNX}\vspace{-10pt}
   \item
     \textbf{Hardware \& EDA Tools}{: NGSPICE, Spectre, Virtuoso, Yosys, OpenROAD, ABC, Mockturtle, Verilator, CocoTB, Design Compiler, PrimeTime}\vspace{-10pt}
   \item
     \textbf{Software \& DevOps}{: Docker, Git, CI/CD, MCP (Model Context Protocol), Claude Code}
  \end{itemize}

% Programming and Development
% Python, Pytorch; Transformers, CNNs; OpenCV, ONNX; VSCode, Git, Linux, ROS
% Hardware and System Design
% RTL: VHDL, SystemVerilog; Image Sensors; Compute-in-Memory; Xilinx FPGAs; ARM V9 Arch., AMBA5
% Tools and Workflow
% IP-XACT, EDA Tools; Full IC Tapeout Flow
% Soft Skills and Miscellaneous
% Task Automator; Great Tutorials Provider; Fantastic PowerPoint Maker



%-----------EDUCATION-----------------
\section{Education}
  \begin{itemize}[leftmargin=*, topsep=0pt, label=$\circ$]
    \resumeEducationSubheading
      {Grenoble Alpes University}{Grenoble, France}
      {Ph.D. in Computer Science}{Apr 2018 -- Apr 2021}
    \resumeEducationSubheading
      {EPFL}{Lausanne, Switzerland}
      % {EPFL | Grenoble Alpes University}{Lausanne, Switzerland | Grenoble, France}
      {"M.Eng. in Electronics (Highest Honors)"}{Sep 2015 -- Sep 2017}
    \resumeEducationSubheading
      {Grenoble Institute of Technology}{Grenoble, France}
      {B.Eng. in Electronics}{Sep 2012 -- Sep 2015}
  \end{itemize}
  \vspace{10pt}



%-------------Patents&Publications------------------

\section{Patents}
\begin{itemize}[leftmargin=*, topsep=0pt, label=$\circ$]
  \begin{small}
    \item M. Bouvier, et al., ``Apparatus, method, and computer program for processing visual event data,'' \href{https://worldwide.espacenet.com/patent/search/family/085781790/publication/WO2024200170A1?q=Maxence%20Bouvier}{\textit{WO2024200170A1}}, 2024.
    \vspace{-10pt}
    \item M. Bouvier, A. Valentian, ``Observation system and associated observation method,'' \href{https://worldwide.espacenet.com/patent/search/family/081324998/publication/US2023196779A1?q=Maxence%20Bouvier}{\textit{US2023196779A1}}, 2023.
    \vspace{-10pt}
    \item F. Carta, et al., ``Pulsing synaptic devices based on phase-change memory to increase the linearity in weight update,'' \href{https://worldwide.espacenet.com/patent/search/family/082100862/publication/US11557343B2?q=Maxence%20Bouvier}{\textit{US11557343B2}}, 2023.
    \vspace{-10pt}
    \item M. Bouvier, A. Bige, ``Device for compensating for movement of an event sensor, and associated systems and methods,'' \href{https://worldwide.espacenet.com/patent/search/family/076730579/publication/WO2022117535A1?q=Maxence%20Bouvier}{\textit{WO2022117535A1}}, 2022.
    \vspace{-10pt}
    \item M. Bouvier, A. Valentian, ``Device for compensating movement of an event-driven sensor and associated observation system and method,'' \href{https://worldwide.espacenet.com/patent/search/family/074668916/publication/US2022101006A1?q=Maxence%20Bouvier}{\textit{US2022101006A1}}, 2022.
  \end{small}
\end{itemize}


\section{Publications}
\begin{itemize}[leftmargin=*, topsep=0pt, label=$\circ$]
  \begin{small}
  \item M. Bouvier, et al., \href{https://maxencebouvier.xyz/files/GENIAL.pdf}{``\textbf{GENIAL: Generative Design Space Exploration via Network Inversion for Low Power Algorithmic Logic Units},''} \textit{Under Review}, 2025.
  \vspace{-7pt}
  \item F. Arnold, M. Bouvier, et al., \href{https://maxencebouvier.xyz/files/IWLS.pdf}{``\textbf{Explicit Sign-Magnitude Encoders Enable Power-Efficient Multipliers},''} \textit{International Workshop on Logic and Synthesis}, 2025.
  \vspace{-7pt}
  \item F. Arnold*, M. Bouvier*, et al., \href{https://arxiv.org/abs/2502.17936}{``\textbf{The Art of Beating the Odds with Predictor-Guided Random Design Space Exploration},''} \textit{62nd ACM/IEEE Design Automation Conference (DAC) (Poster)}, 2025.
  \vspace{-7pt}
  \item C. M. Turrero, M. Bouvier, et al., \href{https://dl.acm.org/doi/abs/10.5555/3692070.3694066}{``\textbf{ALERT-Transformer: Bridging Asynchronous and Synchronous Machine Learning for Real-Time Event-based Spatio-Temporal Data},''} \textit{Proceedings of the 41st ICML}, 2024.
  \vspace{-7pt}
  \item M. Bouvier, et al., \href{https://ieeexplore.ieee.org/document/9586099}{``\textbf{Scalable pitch-constrained neural processing unit for 3D integration with event-based imagers},''} \textit{2021 58th ACM/IEEE Design Automation Conference (DAC)}, 2021.
  \vspace{-7pt}
  \item M. Bouvier, \href{https://theses.hal.science/tel-03405455/}{``\textbf{Study and design of an energy-efficient perception module combining event-based image sensors and spiking neural network with 3D integration technologies},''} \textit{Ph.D. Dissertation, Universit\'e Grenoble Alpes}, 2021.
  \vspace{-7pt}
  \item M. Bouvier, et al., \href{https://dl.acm.org/doi/abs/10.1145/3304103}{``\textbf{Spiking neural networks hardware implementations and challenges: A survey},''} \textit{ACM Journal on Emerging Technologies in Computing Systems (JETC)}, 2019.
  \vspace{-7pt}
  \item P. Vivet, et al., \href{https://ieeexplore.ieee.org/abstract/document/8714886/}{``\textbf{Advanced 3D technologies and architectures for 3D smart image sensors},''} \textit{2019 Design, Automation \& Test in Europe Conference \& Exhibition (DATE)}, 2019.
\end{small}
\end{itemize}


%-------------------------------------------
\end{document}
